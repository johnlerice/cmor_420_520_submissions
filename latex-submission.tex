\documentclass[10pt,letterpaper]{article}
\usepackage[letterpaper, margin=.75in]{geometry}
\usepackage[latin1]{inputenc}
\usepackage{float}
\usepackage{xypic}
\usepackage{graphicx, amsthm, amsmath, amssymb}
\graphicspath{{images/}{../images/}}
\usepackage{mathrsfs }
\usepackage{marginnote}
\usepackage{multicol}
\usepackage{subfiles}
\usepackage{listings}


\usepackage{tikz, pgf, calc}
\usetikzlibrary{arrows,matrix,positioning,fit,calc}
\usepackage[linewidth=1pt]{mdframed}
\usepackage{parskip}
\usepackage{cancel}


%% Theorems %%
\newtheorem{theorem}{Theorem}[section]
\theoremstyle{definition}
\newtheorem{definition}{Definition}[section]
\theoremstyle{definition}
\newtheorem{exmp}{Example}[section]
\newtheorem{prop}{Proposition}[section]
\newtheorem{cor}{Corollary}[section]
\newtheorem{lem}{Lemma}[section]
\newtheorem*{rem}{Remark}

\newcommand{\R}{\mathbb{R}}
\newcommand{\C}{\mathbb{C}}
\newcommand{\N}{\mathbb{N}}
\newcommand{\Q}{\mathbb{Q}}

\newcommand{\sumin}{\sum_{i=1}^n}
\newcommand{\dt}{\Delta t}

\author{jl530}
\title{CMOR 420/520, Homework \#1: \LaTeX \ Submission}

\begin{document}
\maketitle

\section{Communicating with remote repository}
Git command to add, commit, and push file to remote repository
\begin{verbatim}
    #!/bin/bash 
    directory_name=$1 
    git add "$directory_name" 
    git commit -m "Adding $directory_name to remote repository" 
    git push
\end{verbatim}
Output of git log after committing
\begin{verbatim}
    (base) johnle@MacBook-Pro-48 cmor_420_520_submissions % git log --oneline -n 3 
    4d31877 (HEAD -> main, origin/main) Adding HW_1_JohnLe to remote repository 
    9d588fa Initial commit: Adding HW_1_JohnLe and git_push.sh
\end{verbatim}

\section{A script to push a folder to a remote repository}

\begin{lstlisting}[caption={Bash script for pushing a folder to a remote Git repository.}]

    #!/bin/bash 
    directory_name=$1 
    git add "$directory_name" 
    git commit -m "Adding $directory_name to remote repository" 
    git push
\end{lstlisting}

Line-by-line description:
\begin{itemize}
    \item \textbf{Line 1}: specifies the script uses bash
    \item \textbf{Line 2}: allows first argument to be called the directory name
    \item \textbf{Line 3}: runs git add stored in that directory name
    \item \textbf{Line 4}: commits the file with the saved message
    \item \textbf{Line 5}: pushes the file!
\end{itemize}

One way I could see this code not working is if they forget to add a directory, and just went straight to using the shell script. Another thing is that it doesn't check if there is an initialized git repository/directory, if they were to just run this none of the git commands would work. 



\end{document}